\documentclass[10pt,a4paper]{beamer}

\mode<presentation>
{
  %\usetheme{Boadilla}
  \usetheme{Madrid}
  %\usetheme{Singapore}
}


\usepackage[utf8x]{inputenc}
\usepackage{ucs}
\usepackage{amsmath}
\usepackage{amsfonts}
\usepackage{amssymb}
\usepackage{verbatim}
\usepackage{listings}
\usepackage{graphicx} 
\usepackage{longtable}
\usepackage{url}

\lstset{
    basicstyle=\tiny\ttfamily,
    extendedchars=true,
    inputencoding=latin1,
    language=C++,
    keywordstyle=\color{blue},
    identifierstyle=\color{black},
    stringstyle=\color{red},
    showstringspaces=false
}

%\setbeamertemplate{background}{\includegraphics[width=1\textwidth]{natfak_baggrund.pdf}}
\logo{\includegraphics[width=1.5cm]{utfpr_logo2}}
\title{Testes de Unidade com C++}
    \subtitle{Agile Brazil 2011}
    \author{Ramiro Batista da Luz}
    \institute[UTFPR-PPGCA]
    {
      DAINF - Departamento Acadêmico de Informática\\
      Programa de Pós-Graduação em Computação Aplicada\\
      Universidade Tecnológica Federal do Paraná
    }
    \date{26 Junho 2011}
    
%\bibliographystyle{plain}
%\bibliographystyle{cell}
%\bibliographystyle{neuron}
%\bibliographystyle{jtbnew}
%\bibliographystyle{decsci}
\bibliographystyle{abbrvnat}
%\bibliographystyle{plainnat}
%\bibliographystyle{these}
%\bibliographystyle{wmaainf}
%\bibliographystyle{apasoft}

\begin{document}

	\begin{frame}
    	\titlepage
	\end{frame}
  
	\begin{frame}{Quem?}
		Ramiro Batista da Luz, programador, debian-pr \cite{GudPR:01}, grupy-pr \cite{GrupyPR:01}, dojo-pr \cite{DojoPR:01}.
		\begin{itemize}
			\item Mestrando pela Universidade Tecnológica Federal do Paraná.
			\pause
        	\item Programador da Câmara Municipal de Curitiba.
		\end{itemize}  
	\end{frame}

	\begin{frame}{Onde?}
		\begin{itemize}
			\item Nascido em Florianópolis-SC
			\pause
			\item Aos 3 anos foi para Chapecó-SC.
			\pause
			\item Aos 6 anos mudança para Curitiba-PR.
			\pause
			\item Estudou no Positivo, até a graduação, Bacharelado em Informática.
			\pause
			\item Ingressou no mestrado profissional em computação aplicada da UTFPR.
		\end{itemize}
	\end{frame}

	\begin{frame}{Como?}
		\begin{itemize}
			\item Palestra JUnit Adolfo, \cite{Junit:01}
			\pause
			\item Sugestão de Coding Dojo na UTFPR
			\pause
			\item Seleção do Mestrado - Engenharia de Software - Métodos Ágeis - Coding Dojo \cite{CodingDojo:01}.
		\end{itemize}  
	\end{frame}

	\begin{frame}{Porque?}
		\begin{itemize}
			\item Python Unittest \cite{PSF:01}
			\pause
			\item Dissertação - TDD \cite{Beck:2002:TDD:579193} - Coding Dojo \cite{CodingDojo:01}
			\pause
			\item Algoritmos \cite{Ementarios:CABS001} - Testes simples
			\pause
			\item Programação Avançada	 \cite{Ementarios:CABS002} - Foco em orientação a objetos - Testes mais avançados.
		\end{itemize}
	\end{frame}  

	\begin{frame}{Testes}
		\begin{itemize}
			\item TDD - Test Driven Development \cite{Beck:2002:TDD:579193}
			\pause
			\item SUnit - Original para Smalltalk. \cite{wiki:sunit:01} e \cite{wiki:smalltalk:01}. Por Kent Beck. \cite{beck1999kent}
			\pause
			\item xUnit - Nomenclatura usada para generalizar o padrão, muitas linguagens desenvolveram bibliotecas similares, JUnit, NUnit, PyUnit, CPPUnit, ... \cite{WikipediaXUnit:01}
			\pause
			\item Bowling Kata in C por Olve Maudal \cite{OlveMaudal:01} inspirado por Robert C. Martin \cite{RobertCMartin:01}
		\end{itemize}
	\end{frame}  

	\begin{frame}{CPPUnit}
		\paragraph{}
		CPPUnit
		\begin{itemize}
			\item Portado do JUnit para C++ por Michael Feathers \cite{MichaelFeathers:01}
			\pause
			\item Para instalar obter a biblioteca no site \cite{MichaelFeathers:02} ou ...
			\pause
			\item instalar via gerenciador de pacotes(linux/debian) \linebreak
			\# aptitude install libcppunit-1.12-1 libcppunit-dev libcppunit-doc
			\pause
			\item Utilizada na Disciplina de Programação Avançada \cite{Ementarios:CABS001}
		\end{itemize}
	\end{frame}  

	\begin{frame}{CPPUnit Características}
		Características \cite{MichaelFeathers:03}
		\begin{itemize}
			\item Saída XML
			\pause
			\item Saída de texto similar a compilador para integrar com IDE
			\pause
			\item Macros auxiliares para facilitar a declaração da suíte de testes
			\pause
			\item Suporte a preparação de testes hierárquica
			\pause
			\item Registro de testes para reduzir a recompilação
			\pause
			\item Plugin de teste para agilizar o ciclo de compilação/teste
			\pause
			\item Protetor para encapsular a execução do teste
			\pause
			\item Executor de teste MFC (MfcTestRunner \textit{MfcTestRunner}) por Baptiste Lepilleur \cite{BaptisteLepilleur:02}
			\pause
			\item Executor de teste gráfico baseado em QT (QtTestRunner \textit{QtTestRunner})	 por Baptiste Lepilleur \cite{BaptisteLepilleur:01}
		\end{itemize}
	\end{frame}

	\begin{frame}{CPPUnit - Includes}
		\begin{block}{}
				\lstinputlisting[language=C++,linerange=3-8]{TestAcademico.cpp}
		\end{block}
	\end{frame}	

	\begin{frame}{CPPUnit - Saída XML}
		\begin{tiny}
			\begin{block}{}
					\lstinputlisting[language=C++,linerange=28-45]{TestAcademico.cpp}
			\end{block}
		\end{tiny}
	\end{frame}	

	\begin{frame}{CPPUnit - Saída Compiler}
		\begin{tiny}
			\begin{block}{}
					\lstinputlisting[language=C++,linerange=10-26]{TestAcademico.cpp}
			\end{block}
		\end{tiny}
	\end{frame}	

	\begin{frame}{CPPUnit - Macros}
		\begin{tiny}
			\begin{block}{}
					\lstinputlisting[language=C++,linerange=7-23]{TestData.h}
			\end{block}
		\end{tiny}
	\end{frame}	

	\begin{frame}{Exemplo}
		\begin{itemize}
			\item Ver TestAcademico.cpp e TestData.*
		\end{itemize}			
	\end{frame}  

	\begin{frame}{Google Test}
		\paragraph{}
		Google Test
		\begin{itemize}
			\item Utilizado para projetos internos na Google \cite{GoogleTest:01}
			\pause
			\item Para instalar baixar do projeto pelo site \cite{GoogleDownloads:01} ou ...
			\pause
			\item Instalar via gerenciador de pacotes(linux/debian) \linebreak
			\# aptitude install libgtest-dev libgtest0
			\pause
			\item Utilizada para Algoritmos \cite{Ementarios:CABS001} / CodingDojo \cite{DojoPR:01}
		\end{itemize}
	\end{frame}  

	\begin{frame}{Google Test - Características}
		Cartilha ou Como começar? \cite{GoogleTestPrimer:01}
		\begin{itemize}
			\item Permite rodar testes isoladamente
			\pause
			\item Agrupa testes relacionados em casos de teste (Test Cases)
			\pause
			\item Funciona em vários sistemas operacionais e com vários compiladores
			\pause
			\item Não é interrompido nas falhas, permitindo corrigir vários bugs/testes num único ciclo de execução-edição-compilação
			\pause
			\item Automaticamente mantém registro dos testes definidos
			\pause
			\item Permite o reuso de recursos compartilhados entre os testes
		\end{itemize}
	\end{frame}  

	\begin{frame}{Google Test - Rodando testes isolados}
			\begin{block}{}
					\lstinputlisting[language=C++,linerange=1-12]{test_arvore.cpp}
					...
					\lstinputlisting[language=C++,linerange=69-69]{test_arvore.cpp}
			\end{block}
	\end{frame}

	\begin{frame}{Google Test - Test Case}
			\begin{block}{}
					\lstinputlisting[language=C++,linerange=8-17]{test_arvore.cpp}
					...
					\lstinputlisting[language=C++,linerange=349-357]{test_arvore.cpp}
			\end{block}
	\end{frame}

	\begin{frame}[fragile]{Google Test - Não é interrompido}
		\begin{verbatim}
[==========] Running 17 tests from 1 test case.
[----------] Global test environment set-up.
[----------] 17 tests from ArvoreTest
[ RUN      ] ArvoreTest.Insere
test_arvore.cpp:81: Failure
...
[  FAILED  ] ArvoreTest.Insere (7 ms)
[ RUN      ] ArvoreTest.CriaArvoreAltura3
[       OK ] ArvoreTest.CriaArvoreAltura3 (0 ms)
[ RUN      ] ArvoreTest.RemoveNo2FilhosEsquerda
...
		\end{verbatim}
	\end{frame}

	\begin{frame}{Google Test - Registro automático}
			\begin{block}{}
					\lstinputlisting[language=C++,linerange=361-364]{test_arvore.cpp}
			\end{block}
	\end{frame}

	\begin{frame}{Google Test - Reuso}
			\begin{block}{}
					\lstinputlisting[language=C++,linerange=19-21]{test_arvore.cpp}
					...
					\lstinputlisting[language=C++,linerange=30-33]{test_arvore.cpp}
			\end{block}
	\end{frame}

	\begin{frame}{Google Test - Asserções}
			Básicos
			\begin{itemize}
				\item ASSERT\_TRUE(condição); 	 EXPECT\_TRUE(condição);
				\item ASSERT\_FALSE(condição); 	EXPECT\_FALSE(condição);	
			\end{itemize}
	\end{frame}  
	\begin{frame}{Google Test - Comparação Binaria}
			\begin{itemize}
				\item == ASSERT\_EQ(esperado, atual);	EXPECT\_EQ(esperado, atual);
				\item != ASSERT\_NE(val1, val2); 	EXPECT\_NE(val1, val2);
				\item \textless ASSERT\_LT(val1, val2); 	EXPECT\_LT(val1, val2);
				\item \textless= ASSERT\_LE(val1, val2); 	EXPECT\_LE(val1, val2);
				\item \textgreater ASSERT\_GT(val1, val2); 	EXPECT\_GT(val1, val2);
				\item \textgreater= ASSERT\_GE(val1, val2); 	EXPECT\_GE(val1, val2); 
			\end{itemize}
	\end{frame}			
	\begin{frame}{Google Test - Comparação de strings}
			\begin{itemize}
				\item ASSERT\_STREQ(str\_esperada, str\_atual); 	 EXPECT\_STREQ(str\_esperada, str\_atual);
				\item ASSERT\_STRNE(str1, str2); 	EXPECT\_STRNE(str1, str2); 
				\item ASSERT\_STRCASEEQ(str\_esperada, str\_atual);	EXPECT\_STRCASEEQ(str\_esperada, str\_atual);
				\item ASSERT\_STRCASENE(str1, str2);	EXPECT\_STRCASENE(str1, str2);
			\end{itemize}
	\end{frame}			
	\begin{frame}{Google Test - Exceções}
			\begin{itemize}
				\item ASSERT\_THROW(comando, tipo\_exceção); 	 EXPECT\_THROW(comando, tipo\_exceção);
				\item ASSERT\_ANY\_THROW(comando); 	EXPECT\_ANY\_THROW(comando);
				\item ASSERT\_NO\_THROW(comando); 	EXPECT\_NO\_THROW(comando); 	
			\end{itemize}
	\end{frame}			
	\begin{frame}{Google Test - Comparação de números ponto flutuante}
			\begin{itemize}
				\item ASSERT\_FLOAT\_EQ(esperado, atual); 	 EXPECT\_FLOAT\_EQ(esperado, atual);
				\item ASSERT\_DOUBLE\_EQ(esperado, atual); 	EXPECT\_DOUBLE\_EQ(esperado, atual); 	
			\end{itemize}
	\end{frame}  

	\begin{frame}{Exemplo}
		\begin{itemize}
			\item Ver arvore.cpp e test\_arvore.cpp
		\end{itemize}			
	\end{frame}  

	\begin{frame}{Experimentando}
		\begin{itemize}
			\item MiniDojo
		\end{itemize}
	\end{frame}  

	\begin{frame}{Agradecimentos}
		\begin{itemize}
			\item Carlos Niemeyer - Diretor de Informática e João Claudio Derosso - Presidente da Câmara Municipal de Curitiba.
			\pause
			\item Adolfo Gustavo Serra Seca Neto - Orientador
			\pause
			\item Tania Mezzadri - Algoritmos
			\pause
			\item Jean Simão e João Alberto Fabro - Programação Avançada
			\pause
			\item Henrique Pereira(@ikkebr) e Gabriel Oliveira(@GpaOliveira)
			\pause
			\item Arthur Furlan(@afurlan) Autor do http://va.mu
			\pause
			\item Organizadores Agile Brazil
			\pause
			\item Ao público presente.
		\end{itemize}
		
	\end{frame}  

  \begin{frame}[allowframebreaks]{Referencias} 
   	\bibliography{apresentacao}
  \end{frame}  

\end{document}
